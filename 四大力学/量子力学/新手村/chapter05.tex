\chapter{Identical Particles}

\section{Two-Particle System}
\begin{itemize}
	\item \(\Psi(\mathbf{r}_1,\mathbf{r}_2,t)\)
	\item \(\displaystyle{ i\hbar\pdv{\Psi}{t}=\hat{H}\Psi,\qquad\text{where} \quad \hat{H}=-\frac{\hbar^2}{2m_1}\nabla_1^2-\frac{\hbar^2}{2m_2}\nabla_2^2+V(\mathbf{r}_1,\mathbf{r}_2,t) }\)
	\item \(\displaystyle{  \int |\Psi(\mathbf{r}_1,\mathbf{r}_2,t)|^2 
		\dd^3\mathbf{r}_1 \dd^3\mathbf{r}_2= 1 }\)
	\item \(\Psi(\mathbf{r}_1,\mathbf{r}_2,t)=\psi(\mathbf{r}_1,\mathbf{r}_2)e^{-iEt/\hbar}\)
	\item \(\displaystyle{  -\frac{\hbar^2}{2m_1}\nabla_1^2\psi -\frac{\hbar^2}{2m_2}\nabla_2^2\psi+V\psi=E\psi }\)
\end{itemize}
\begin{enumerate}
	\item Nointeracting Particles
	\[\Psi(\mathbf{r}_1,\mathbf{r}_2,t) = \Psi_a(\mathbf{r}_1,t) + \Psi_a(\mathbf{r}_2,t)\]
    \item Central potential (helium atom)
	   \[V(\mathbf{r}_1,\mathbf{r}_2)=\frac{1}{4\pi\epsilon_0}
	     \qty(-\frac{2e^2}{|\mathbf{r}_1|}-\frac{2e^2}{|\mathbf{r}_2|}+\frac{e^2}{|\mathbf{r}_1-\mathbf{r}_2|})\]
\end{enumerate}
\subsection{Bosons and Fermions}
\begin{enumerate}
	\item Bosons state
	\[\psi_+(\mathbf{r}_1,\mathbf{r}_2)=A[\psi_a(\mathbf{r}_1)\psi_b(\mathbf{r}_2 )
	                                       +\psi_b(\mathbf{r}_1)\psi_a(\mathbf{r}_2) ]\qquad A=\frac{1}{\sqrt{2}}\]
	\begin{itemize}
		\item symmetric under interchange: \(\psi_+(\mathbf{r}_1,\mathbf{r}_2)=\psi_+(\mathbf{r}_2,\mathbf{r}_1)\)
        \item all particles with integer spin are bosons
	\end{itemize}								   
    \item Fermions state
    \[\psi_-(\mathbf{r}_1,\mathbf{r}_2)=A[\psi_a(\mathbf{r}_1)\psi_b(\mathbf{r}_2 )
	                                       -\psi_b(\mathbf{r}_1)\psi_a(\mathbf{r}_2 )]\qquad A=\frac{1}{\sqrt{2}}\]
	\begin{itemize}
        \item symmetric under interchange: \(\psi_-(\mathbf{r}_1,\mathbf{r}_2)=-\psi_-(\mathbf{r}_2,\mathbf{r}_1)\)
		\item all particles with half integer spin are fermions
  \item Pauli exclusion principle: two identical fermions cannot occupy the same state
  \[\psi_-(\mathbf{r}_1,\mathbf{r}_2)=A[\psi_a(\mathbf{r}_1)\psi_a(\mathbf{r}_2 )
	                                       -\psi_a(\mathbf{r}_1)\psi_a(\mathbf{r}_2 )]=0\]
	\end{itemize}	
\end{enumerate}
\subsection{Exchange Forces}
Suppose \(\psi(x_1,x_2)=\psi_a(x_1)\psi_b(x_2)\)\newline
calculate \(\expval{(x_1-x_2)^2} = \expval{x_1^2}+\expval{x_2^2}-2\expval{x_1x_2}\)
\begin{enumerate}
	\item Distinguishable particles \newline
         \[ \expval{(x_1-x_2)^2}_d = \expval{x^2}_a+ \expval{x^2}_b+  2\expval{x}_a \expval{x}_b\]
 \item Identical particles
        \[\expval{x_1^2}=\expval{x_2^2}=\frac{1}{2}\qty(\expval{x^2}_a+\expval{x^2}_b)\]
		\[\expval{x_1x_2} = \expval{x}_a\expval{x}_b\pm |\expval{x}_{ab}|^2\]
		 where \[\expval{x}_{ab}\equiv\int x\psi_a(x)^*\psi_b(x)\dd x\]
		 thus \[\expval{(x_1-x_2)^2}_{\pm} = \expval{x^2}_a+\expval{x^2}_b +2\expval{x}_a\expval{x}_b \mp 2|\expval{x}_{ab}|^2\]
\end{enumerate}
\[\Rightarrow\qquad\expval{(\Delta x )^2}_\pm =\expval{(\Delta )^2}_d\mp 2|\expval{x}_{ab}|^2 \]
Exchange force: (if \(\expval{x}_{ab}\neq 0\))
\begin{itemize}
	\item force of attraction between identical bosons
 \item force of repulsion between identical fermions
\end{itemize}
\subsection{Spin}
Pauli principle: two electrons in a given position state as long as their spins are in the singlet configuration
\[\psi(\mathbf{r}_1,\mathbf{r}_2)\chi(1,2)=-\psi(\mathbf{r}_2,\mathbf{r}_1)\chi(2,1)\]

\subsection{Generlized Symmetrization Principle}
general statement, if you have n identical particles
\[\ket{(1,2,\cdots,i,\cdots,j,\cdots,n)} = \pm \ket{(1,2,\cdots,j,\cdots,i,\cdots,n)}\]
\section{Atoms}
\[\hat{H}=\sum_{j=1}^Z\qty{ - \frac{\hbar^2}{2m}\nabla^2_j - 
                \qty(\frac{1}{4\pi\epsilon_0})\frac{Ze^2}{r_j}} 
				   + \frac{1}{2}\qty(\frac{1}{4\pi\epsilon_0})\sum_{j\neq k}^Z\frac{e^2}{|\mathbf{r}_j - \mathbf{r}_j|}\]
\begin{itemize}
	\item \(Z\): atomic number
 \item \(Ze\): electric charge
 \item in curly brackets: kinetic plus potential energy of the \(j\)th electron
 \item the second sum: the potential energy associated with the mutual repulsion of the electrons
\end{itemize}
\subsection{Helium}
\[\hat{H}=\qty{ - \frac{\hbar^2}{2m}\nabla^2_1 - 
                \qty(\frac{1}{4\pi\epsilon_0})\frac{2e^2}{r_1}} + \qty{ - \frac{\hbar^2}{2m}\nabla^2_2 - 
                \qty(\frac{1}{4\pi\epsilon_0})\frac{2e^2}{r_2}} 
				   + \frac{1}{4\pi\epsilon_0} \frac{e^2}{|\mathbf{r}_1 - \mathbf{r}_2|}\]
ignore the last term 
\[\psi(\mathbf{r}_1,\mathbf{r}_2)=\psi_{n\ell m}(\mathbf{r}_1)\psi_{n'\ell' m'}(\mathbf{r}_2)\qquad\text{with}\quad E=4(E_n+E_{n'})\]
ground state
\[\psi_0(\mathbf{r}_1,\mathbf{r}_2)=\psi_{100}(\mathbf{r}_1)\psi_{100}(\mathbf{r}_2)=\frac{8}{\pi a^3}{\rm e}^{-2(r_1+r_2)/a}
\qquad\text{with}\quad E=8(-13.6\,{\rm eV})=-109\,{\rm eV}\]
\begin{itemize}
	\item symmetric function
 \item singlet
\end{itemize}
\subsection{The Periodic Table}
Hund's rules

\[^{2S+1}L_J\]
\section{Solids}

\subsection{The Free Elctron Gas}
Suppose
\[V(x,y,z) = \begin{cases}
	0 ,\qquad & 0<x<l_x,0<y<l_y,0<z<l_z\\
	\infty & otherwise
\end{cases}\] 
Wave functions are 
\[\psi_{n_x,n_y,n_z} = \sqrt{\frac{8}{l_xl_yl_z}}\sin\qty(\frac{n_x\pi}{l_x}x)\sin\qty(\frac{n_y\pi}{l_y}y)\sin\qty(\frac{n_z\pi}{l_z}z)\]
energies are 
\[E_{n_x,n_y,n_z} =\frac{\hbar^2\pi^2}{2m}\qty(\frac{n_x^2}{l_x^2} + \frac{n_y^2}{l_y^2} + \frac{n_z^2}{l_z^2})=\frac{\hbar^2k^2}{2m}\qquad \mathbf{k}\equiv(k_x,k_y,k_z)\]

\subsection{Band Structure}
Bloch's therem 
 \[V(x+a)=V(x) \qquad\rightarrow\qquad \psi(x+a)={\rm e}^{iqa}\psi(x)\]
Dirac comb
\[V(x) = \alpha\sum_{j=0}^{N-1}\delta(x-ja) \]
